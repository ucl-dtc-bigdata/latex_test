\documentclass[a4paper,onecolumn,11pt]{article}

\usepackage[left=2.5cm,top=2.5cm,bottom=2.5cm,right=2.5cm]{geometry}
\usepackage{hyperref}
\usepackage[utf8]{inputenc}
\usepackage{graphicx}
\usepackage{amsmath,amssymb}
\usepackage[numbers,elide,sort&compress]{natbib}
\usepackage{eurosym}
\usepackage{enumitem}
%\setitemize{noitemsep,topsep=0pt,parsep=0pt,partopsep=0pt,leftmargin=*}

\usepackage{fancyhdr}
\pagestyle{fancy}
%\renewcommand{\headrulewidth}{0pt} % Remove line at top
 
\lhead{STFC CDT Big Data}
\chead{University College London}
\rhead{\today}
\cfoot{\thepage}

\begin{document}

\section*{UCL's research strength in the scientific areas to be addressed by the centre (2.5 pages)}

 FFFFFFFFFFFFFFFFFF dolor sit amet, consectetur adipiscing elit. Integer consequat risus sit amet mi ultrices at pulvinar augue ultricies. Ut pulvinar nulla vitae ante tincidunt a viverra est vehicula. Duis a tortor mi. Sed metus orci, cursus ac egestas id, egestas sit amet ipsum. Ut rhoncus volutpat tortor, sed mollis elit mollis in. Aliquam vel risus in neque rutrum accumsan. Quisque egestas velit eu tellus ullamcorper nec vulputate justo convallis. Nam posuere odio ac diam egestas mollis. Suspendisse at libero a eros placerat tristique in at mi. Nullam bibendum bibendum nisi, in consectetur tellus tempus id. Praesent porta diam in nisl eleifend imperdiet. Integer volutpat tempus dui.

 Morbi feugiat urna quis felis dictum tempus. Suspendisse potenti. Fusce orci arcu, imperdiet nec eleifend vitae, sagittis vitae orci. Etiam eget urna leo. Morbi auctor tellus quis risus dapibus pulvinar~\citep{Mauersberger:1986}.

 The aims of the CDT: 
\begin{itemize}
\item UCL will provide training for 4 years
\item This will cost \pounds 95,000 per student
\end{itemize}

%\bibliographystyle{plain}
%\bibliography{my-bib.bib}
\begin{thebibliography}{1}
\bibitem{Mauersberger:1986}
R. Mauersberger, T. L. Wilson, and C. Henkel,
``The discovery of an interstellar $^{15}$NH$_3$ maser,''
Astron. Astrophys. {\bf 160}, L13-L16 (1986)
\end{thebibliography}

\end{document}
